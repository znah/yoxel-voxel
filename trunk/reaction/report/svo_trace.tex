\section{Анализ производетельности трассировки лучей в воксельном октарном дереве}

В данном разделе представленны некоторые наблюдения влияния различных аспектов реализации и параметров трассировщика на производительность трассировки лучей в октарном дереве, реализованном при помощи технологии CUDA. Отметим, что производительность достаточно сильно зависит от сложности геометрии, поподающей в поле зрения наблюдателя, поэтому все замеры производительности производились при визуализации одного и того же участка сцены, который изображен на рис.~\ref{fig:trace_scene}.

\begin{figure}[h]
\center
\includegraphics[width=\textwidth]{trace.jpg}
\caption{Тестовая сцена}
\label{fig:trace_scene}
\end{figure}

Основное внимание в этом разделе уделяется вопросам балансировки вычислительной нагрузки и эффективной работе с памятью GPU.

\subsection{Балансировка вычислительной нагрузки}

Заметим, что количество итераций алгоритма трассировки на пиксель может сильно различатся для лучей, порожденных разными пикселями. На рис.~\ref{fig:trace_iters} показано количество итераций аглоритма трассировки, требуемое для каждого пикселя изображения. Среднее количество итераций близко к 50, однако, для отдельных пикселей изображения может достигать нескольких сотен. На рис.~\ref{fig:trace_iter_distrib} показано распределение количества итераций на луч.

На каждой итерации осуществляется одно из следующих действий:

\begin{itemize}
  \item Спуск вниз по октарному дереву на один уровень.
  \item Переход к следующему пересекаемому лучем узлу дерева, с возможным подьемом на несколько уровней.
  \item Сохранение результата при пересечении луча с непустым вокселем.
\end{itemize}

\begin{figure}[h]
\center
\includegraphics[width=\textwidth]{trace_iters.jpg}
\caption{Количество итераций алогритма трассировки на пиксель}
\label{fig:trace_iters}
\end{figure}

\begin{figure}[h]
\center
\includegraphics[width=\textwidth]{ballance/trace_iter_distrib.pdf}
\caption{Распределение количества итераций на луч}
\label{fig:trace_iter_distrib}
\end{figure}

В большинстве областей изображения количество итераций на пиксель достаточно однородно, однако, в местах, где лучи проходят по касательной к поверхности, возникают разрывы во времени работы алгоритма для соседних пикселей. Так, группы соседних пикселей трассируются потоками, лежащими в однов warp'е, это может првести в простою потоков, завершивших трассировку, в ожидание из более медленных соседей. Механизм возникновения этого явления показан на рис.~\ref{fig:warp_loads}.

\begin{figure}[h]
\center
\includegraphics[width=\textwidth]{ballance/warp_loads.pdf}
\caption{Различные варианты заполнения warp'ов при визуализации тестовой сцены}
\label{fig:warp_loads}
\end{figure}

В случае когеррентных потоков, среднее заполнение warp'ов составило около 81\%. Был проведен эксперимент с некогеррентными лучами. В данном эксперименте была выполнена визуализация тестовой сцены, однако задания для потоков вычисления были перемешаны таким образом, что каждый warp состоял из потоков, обрабатывавщих пиксели лежавшие в произвольных случайных частях изображения. В этом случает оффективность заполнения warp'ов снижалась до 47\%, а производительность падала более чем в два раза. Данные о производтельности трассировщика лучей при отрисовке тестовой сцены в разрешении 1024х768 представлены в таблице \ref{tab:coher_perf}.

\begin{table}[ht]
\center
\label{tab:coher_perf}
\begin{tabular}{l|p{.2\textwidth}|p{.2\textwidth}|p{.2\textwidth}}
 Лучи  & Загруженность warp'ов & Время трассировки GeForce 8800 GTS (мс) &  Время трассировки GeForce GTX 275 (мс) \\
\cline{1-4}
  Когерентные  & 81\% & 71 & 35 \\
  Некогерентные & 47\% & 143 &  72 \\
\end{tabular}
\caption{Сравнение производительности трассировщика лучей в случае когеррентных и некогеррентных лучей}
\end{table}


\subsection{Особенности работы с памятью и их влияние на производительность}

В данной главе изложены некоторые наблюдения, полученные в ходе работ над визуализацией воксельной сцены, представленной ввиде SVO.

Большинство экспериметров с варьированием параметров трассировщика проведено на сцене, показанной на рис.~\ref{fig:trace}. Время работы текущей версии GPU трассировщика лучей при построении данного изображения в разрешении 1024x768 составляет около 34~мс. 

\begin{figure}[h]
\center
\includegraphics[width=\textwidth]{trace_res_time.pdf}
\caption{Зависимость времени трассировки от разрешения}
\label{fig:trace_res_time}
\end{figure}

\begin{figure}[h]
\center
\includegraphics[width=\textwidth]{trace_lod_time.pdf}
\caption{Зависимость времени трассировки глубины дерева}
\label{fig:trace_lod_time}
\end{figure}


\begin{figure}[h]
\center
\includegraphics[width=\textwidth]{trace_scheduler.pdf}
\caption{Диаграмма исполнения для когерентных и перемешанных потоков}
\label{fig:trace_scheduler}
\end{figure}
