\section*{Приложение A. Фрагментный шейдер, используемый при вокселизации}
\addcontentsline{toc}{section}{Приложение A. Фрагментный шейдер, используемый при вокселизации}
\label{apx:voxelization_code}
\begin{lstlisting}[caption={Фрагментный шейдер, используемый при вокселизации}]
uniform usampler1D columnBits;
uniform float sliceNum;
          
struct PixelOutput
{
  uint4 s0 : COLOR0;
  uint4 s1 : COLOR1;
  uint4 s2 : COLOR2;
  uint4 s3 : COLOR3;
  uint4 s4 : COLOR4;
  uint4 s5 : COLOR5;
  uint4 s6 : COLOR6;
  uint4 s7 : COLOR7;
};

uint4 slice(float z0, uint bits)
{
  const uint4 zeros = uint4(0);
  const uint4 ones  = uint4(0xffffffff);
  if (z < z0)
    return zeros;
  if (z < z0 + 1.0f)
    return bits;
  return ones;
}

PixelOutput main(float3 pos : TEXCOORD0)
{
  PixelOutput output;
            
  const float sliceDepth = 128.0f;
  const float halfVoxel = 0.5f / sliceDepth;
  float z = pos.z * sliceNum - halfVoxel;
  uint4 bits = tex1D( columnBits, frac(z) );
            
  output.s0 = slice(0.0f, bits);
  output.s1 = slice(1.0f, bits);
  output.s2 = slice(2.0f, bits);
  output.s3 = slice(3.0f, bits);
  output.s4 = slice(4.0f, bits);
  output.s5 = slice(5.0f, bits);
  output.s6 = slice(6.0f, bits);
  output.s7 = slice(7.0f, bits);
  return output;
}
\end{lstlisting}
